%%%%%%%%%%%%%%%%%
% This is an sample CV template created using altacv.cls
% (v1.7.2, 28 Aug 2024) written by LianTze Lim (liantze@gmail.com), based on the
% CV created by BusinessInsider at http://www.businessinsider.my/a-sample-resume-for-marissa-mayer-2016-7/?r=US&IR=T
%
%% It may be distributed and/or modified under the
%% conditions of the LaTeX Project Public License, either version 1.3
%% of this license or (at your option) any later version.
%% The latest version of this license is in
%%    http://www.latex-project.org/lppl.txt
%% and version 1.3 or later is part of all distributions of LaTeX
%% version 2003/12/01 or later.
%%%%%%%%%%%%%%%%

%% Use the "normalphoto" option if you want a normal photo instead of cropped to a circle
% \documentclass[10pt,a4paper,normalphoto]{altacv}

\documentclass[10pt,a4paper,ragged2e,withhyper]{altacv}
%% AltaCV uses the fontawesome5 and simpleicons packages.
%% See http://texdoc.net/pkg/fontawesome5 and http://texdoc.net/pkg/simpleicons for full list of symbols.

% Change the page layout if you need to
\geometry{left=1.25cm,right=1.25cm,top=1.5cm,bottom=1.5cm,columnsep=1.2cm}

% The paracol package lets you typeset columns of text in parallel
\usepackage{paracol}

\iftutex
  % If using xelatex or lualatex:
  \setmainfont{Lato}
\else
  % If using pdflatex:
  \usepackage[default]{lato}
\fi


% Change the colours if you want to
\definecolor{VividPurple}{HTML}{0277BD}
\definecolor{SlateGrey}{HTML}{2E2E2E}
\definecolor{LightGrey}{HTML}{666666}
% \colorlet{name}{black}
\colorlet{tagline}{VividPurple}
\colorlet{heading}{VividPurple}
\colorlet{headingrule}{VividPurple}
% \colorlet{subheading}{PastelRed}
\colorlet{accent}{VividPurple}
\colorlet{emphasis}{SlateGrey}
\colorlet{body}{LightGrey}

% Change some fonts, if necessary
% \renewcommand{\namefont}{\Huge\rmfamily\bfseries}
% \renewcommand{\personalinfofont}{\footnotesize}
% \renewcommand{\cvsectionfont}{\LARGE\rmfamily\bfseries}
% \renewcommand{\cvsubsectionfont}{\large\bfseries}

% Change the bullets for itemize and rating marker
% for \cvskill if you want to
\renewcommand{\cvItemMarker}{{\small\textbullet}}
\renewcommand{\cvRatingMarker}{\faCircle}
% ...and the markers for the date/location for \cvevent
% \renewcommand{\cvDateMarker}{\faCalendar*[regular]}
% \renewcommand{\cvLocationMarker}{\faMapMarker*}


% If your CV/résumé is in a language other than English,
% then you probably want to change these so that when you
% copy-paste from the PDF or run pdftotext, the location
% and date marker icons for \cvevent will paste as correct
% translations. For example Spanish:
% \renewcommand{\locationname}{Ubicación}
% \renewcommand{\datename}{Fecha}


%% Use (and optionally edit if necessary) this .tex if you
%% want to use an author-year reference style like APA(6)
%% for your publication list
% When using APA6 if you need more author names to be listed
% because you're e.g. the 12th author, add apamaxprtauth=12
\usepackage[backend=biber,style=apa6,sorting=ydnt]{biblatex}
\defbibheading{pubtype}{\cvsubsection{#1}}
\renewcommand{\bibsetup}{\vspace*{-\baselineskip}}
\AtEveryBibitem{%
  \makebox[\bibhang][l]{\itemmarker}%
  \iffieldundef{doi}{}{\clearfield{url}}%
}
\setlength{\bibitemsep}{0.25\baselineskip}
\setlength{\bibhang}{1.25em}


%% Use (and optionally edit if necessary) this .tex if you
%% want an originally numerical reference style like IEEE
%% for your publication list

%% sample.bib contains your publications
\addbibresource{anna.bib}

\begin{document}
\name{Anna Szentirmay}
\tagline{Übersetzerin}
% Cropped to square from https://en.wikipedia.org/wiki/Marissa_Mayer#/media/File:Marissa_Mayer_May_2014_(cropped).jpg, CC-BY 2.0
%% You can add multiple photos on the left or right
\photoR{2.5cm}{anna.jpg}
% \photoL{2cm}{Yacht_High,Suitcase_High}
\personalinfo{%
  % Not all of these are required!
  % You can add your own with \printinfo{symbol}{detail}
  \email{szentirmayanna@gmail.com}
  \phone{+49 176 63264458}
  \mailaddress{Graunstr 11, 13355 Berlin}
  % \location{Germany}
  % \homepage{marissamayr.tumblr.com}
  % \twitter{@marissamayer}
  % \xtwitter{@marissamayer}
  %\linkedin{adam-vincze-ab60b76}
  %\github{github.com/advincze} % I'm just making this up though.
  % \orcid{0000-0000-0000-0000} % Obviously making this up too.
  %% You can add your own arbitrary detail with
  %% \printinfo{symbol}{detail}[optional hyperlink prefix]
  % \printinfo{\faPaw}{Hey ho!}
  %% Or you can declare your own field with
  %% \NewInfoFiled{fieldname}{symbol}[optional hyperlink prefix] and use it:
  % \NewInfoField{gitlab}{\faGitlab}[https://gitlab.com/]
  % \gitlab{your_id}
	%%
  %% For services and platforms like Mastodon where there isn't a
  %% straightforward relation between the user ID/nickname and the hyperlink,
  %% you can use \printinfo directly e.g.
  % \printinfo{\faMastodon}{@username@instace}[https://instance.url/@username]
  %% But if you absolutely want to create new dedicated info fields for
  %% such platforms, then use \NewInfoField* with a star:
  % \NewInfoField*{mastodon}{\faMastodon}
  %% then you can use \mastodon, with TWO arguments where the 2nd argument is
  %% the full hyperlink.
  % \mastodon{@username@instance}{https://instance.url/@username}
}

\makecvheader

%% Depending on your tastes, you may want to make fonts of itemize environments slightly smaller
\AtBeginEnvironment{itemize}{\small}

%% Set the left/right column width ratio to 6:4.
\columnratio{0.6}

% Start a 2-column paracol. Both the left and right columns will automatically
% break across pages if things get too long.
\begin{paracol}{2}

\cvsection{Erfahrung}

\cvevent{Übersetzerin}{Selbstständig}{Jan 2024 -- heute}{Berlin}{}
\begin{itemize}
  \item Übersetzung von Kinderbüchern aus dem Deutschen ins Ungarische.
 % \item Developed data quality and ownership processes for self-service access to the data warehouse.
  % \item Measured and optimized the \textbf{sales and marketing funnel}, increasing win rates, improving stage conversion, and shortening the sales cycle.
\end{itemize}

\divider
% \medskip

\cvevent{Team Koordinatorin in Recruiting}{KWS Gruppe}{2021 -- heute}{Berlin}{}
\begin{itemize}
  \item Leitung eines Teams von zehn Recruitern
  %\item Led the technical roadmap and aligned with stakeholders to identify key priorities
  %\item Evolved the data platform into a decentralized \textbf{Data Mesh}, driving innovation across the company
  %\item Established data governance and data quality processes to ensure ownership, reduce risk, and improve data adoption.
  %\item Implemented governance for a central \textbf{Kafka} cluster, enabling an event-driven microservice architecture used by 10+ development teams.
\end{itemize}

\divider
% \medskip

\cvevent{HR Managerin}{Candena GmbH}{2020 -- 2021}{Berlin}{}
\begin{itemize}
  \item Betreuung aller HR-Prozesse im Unternehmen
  % \item Data Warehouse for Zalon to enable Analytics and ML development. 
  %\item Developed an MLOps platform to streamline data science projects.
  %\item Created a DMP for Zalando Media solutions for segmentation and Ad targeting based on user attributes and behavior.
\end{itemize}


% \divider
\medskip

\cvevent{SEA und Country Managerin}{Visual Meta GmbH}{2013 -- 2019}{Berlin}{}
\begin{itemize}
  \item Leitung des ungarischen Teams von shopalike.hu
  % \item Establishing DevOps best practices to improve the quality of deployments and the velocity of development.
  % \item New presentation layer, Mobile products and CMS piloting tools for WELT.de with improved maintenance and perfomance
\end{itemize}

% \divider
\medskip

\cvevent{HR Managerin}{Found Fair Ventures GmbH}{2011 -- 2013}{Berlin}{}
\begin{itemize}
  \item Unterstützung von HR-Prozesse
\end{itemize}

% \divider

\cvevent{Kundensupport und Teamleitung}{Arvato Services}{2008 -- 2011}{Münster, Potsdam}{}
\begin{itemize}
  \item Kundensupport und Teamleitung für verschiedene internationale Unternehmen
\end{itemize}


\cvsection{Bildung}

\cvevent{Post.grad in Kinder- und Jugendliteratur}{}{2022 -- 2023}{Universität Gáspár Károli}{Budapest, Ungarn}
\cvevent{Master in Kommunikation und portugiesischer Sprache und Literatur}{}{2002 -- 2009}{Universität Loránd Eötvös}{Budapest, Ungarn}

%\cvsection{Skills}

% Adapted from @Jake's answer from http://tex.stackexchange.com/a/82729/226
% \wheelchart{outer radius}{inner radius}{
% comma-separated list of value/text width/color/detail}
% Some ad-hoc tweaking to adjust the labels so that they don't overlap
%\hspace*{-1em}  %% quick hack to move the wheelchart a bit left
% \wheelchart{1.5cm}{0.5cm}{%
  % 30/9em/accent/Data Engineering,
  % 5/9em/accent!20/Data Strategy,
  % 10/9em/accent/People Management,
  % 20/9em/accent!60/Leadership,
  % 5/9em/accent!40/Agile Practices,
  % 10/8em/accent!40/Analytics,
  % 1/9em/accent!20/Machine Learning,
  % 10/9em/accent!60/\footnotesize\\[1ex]Software Engineering \& Architecture
  % }

% use ONLY \newpage if you want to force a page break for
% ONLY the currentc column
% \newpage

\cvsection{Publikationen}

% %% Specify your last name(s) and first name(s) as given in the .bib to automatically bold your own name in the publications list.
% %% One caveat: You need to write \bibnamedelima where there's a space in your name for this to work properly; or write \bibnamedelimi if you use initials in the .bib
% %% You can specify multiple names, especially if you have changed your name or if you need to highlight multiple authors.
\mynames{Lim/Lian\bibnamedelima Tze,
  Wong/Lian\bibnamedelima Tze,
  Lim/Tracy,
  Lim/L.\bibnamedelimi T.}
% %% MAKE SURE THERE IS NO SPACE AFTER THE FINAL NAME IN YOUR \mynames LIST

\nocite{*}

\printbibliography[heading=pubtype,title={\printinfo{\faBook}{Bücher}},type=book]

% \divider

% \printbibliography[heading=pubtype,title={\printinfo{\faFile*[regular]}{Journal Articles}}, type=article]

% \divider

% \printbibliography[heading=pubtype,title={\printinfo{\faUsers}{Conference Proceedings}},type=inproceedings]

%% Switch to the right column. This will now automatically move to the second
%% page if the content is too long.
\switchcolumn

\cvsection{Über mich}

\cvachievement{\faLightbulb}{Motivation}{Ich nutze jede Gelegenheit, um meinen Horizont zu erweitern und genieße besonders herausfordernde Texte mit spielerischer Sprache.}

% \divider

\cvachievement{\faUserFriends}{Netzwerk}{
  Ich genieße es, mit anderen Menschen im Bereich der Kinderliteratur in Kontakt zu treten.
}

% \divider

\cvachievement{\faDraftingCompass}{Mission}{
  Leidenschaftlich daran interessiert, bedeutungsvolle Geschichten zu erzählen und jungen Lesern zugänglich zu machen.
}

% \divider

\cvachievement{\faUser}{Privat}{
  Seit 2009 lebe ich in Berlin. Ich habe zwei Töchter, die meine treuesten Leserinnen sind.
}

% \divider

% \cvachievement{\faTruck}{Deliver today}{
%  Perfect is the enemy of done. I believe bias for action, quick iterations and customer feedback gets to the right solutions.
%}

%\cvsection{Programming}

%\cvskill{SQL}{4.5} %% supports X.5 values.
%\cvskill{Python}{4}
%\cvskill{Go(lang)}{4}

%\cvsection{Technologies}

% \cvtag{Python}
% \cvtag{Go(lang)}
% \cvtag{SQL}

% \smallskip

%\cvtag{AWS}
%\cvtag{Snowflake}
%\cvtag{Redshift}
%\cvtag{DuckDB}
%\cvtag{Superset}
%\cvtag{Metabase}

%\smallskip

%\cvtag{Terraform}
%\cvtag{DBT}
%\cvtag{Spark}
%\cvtag{Kubernetes}

%\cvsection{Strengths}

%\cvtag{Enthusiastic}
%\cvtag{Motivator}
%\cvtag{Leader}
%\cvtag{Builder}

%\smallskip


%\cvtag{Cloud Infrastructure}
%\cvtag{IaC}
%\cvtag{Data Gov.}
%\cvtag{CI/CD}
%\cvtag{DevOps}
%\cvtag{Data Modeling}
%\cvtag{DWH}
%\cvtag{Data Mesh}
%\cvtag{ETL}
%\cvtag{Data Quality}
%\cvtag{Microservices}

\cvsection{Sprachen}
\cvtag{Ungarisch}
\cvtag{Deutsch}
\cvtag{Englisch}



% \newpage

% \cvsection{Referees}

% % \cvref{name}{email}{mailing address}
% \cvref{Prof.\ Alpha Beta}{Institute}{a.beta@university.edu}
% {Address Line 1\\Address line 2}

% \divider

% \cvref{Prof.\ Gamma Delta}{Institute}{g.delta@university.edu}
% {Address Line 1\\Address line 2}

\end{paracol}

\end{document}
